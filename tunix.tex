\documentstyle[12pt]{article}

\pagestyle{empty}
\parindent0mm   
\topmargin0cm
\textwidth16cm \textheight23cm \oddsidemargin0mm
\newcommand{\aaa}{\vspace{2.5ex plus0.3ex minus0.3ex}}
%-----------------------------------------------------------------------------
\begin{document}

\begin{center}
{\huge Notes on Tunix}
\end{center}

This is not the technical documentation for the
software Tunix. The documentation is to be found
in the book
\begin{quote}
R. M. Switzer, {\it Operating Systems: A Practical
Approach}, Prentice Hall, 1993
\end{quote}
This paper is merely a set of notes on the installation
and use of Tunix.

\section{Platform}
\noindent

Tunix, as delivered, is executable on an Intel processor
80386 or 80486 under Unix System V, Release 3 and
upwards.


The part that makes porting to other
architectures non-trivial is the very short but
somewhat tricky assembly module {\tt context.s}.
This module is responsible for manipulating the
stack to make a context switch between threads
possible. In order to port Tunix to other architectures
you will need to rewrite this assembler module. That
means, of course, that you will need to understand
the stack conventions of your local C compiler and you
will need to be conversant with writing assembler
code on your machine.


Under SunOS it is conceivable these manipulations in
assembler could be avoided entirely by using the
threads provided to programmers by the operating
system. If you want to try something like this,
take a look at the C module {\tt thread.c}, which
implements the Tunix threads.


\section{Installing Tunix}
\noindent

You have presumably already unpacked the tar-file
{\tt tunix.tar} for otherwise you would not be reading
this text. In the same directory in which you found
this text you will find a subdirectory {\tt os} and there
you will find a shell script  called
{\tt build\_os}. In principle all you have to do is
enter the command
\begin{verbatim}
    build_os
\end{verbatim}
The script will automatically build all the parts
of Tunix and deposit the executable binaries in
the subdirectory {\tt os/result}. However, you
would be well advised to watch for error messages
that might occur while the script is working --
or perhaps, better still, redirect the output to
a file to be inspected when the script is finished.

\section{Populating the file system}
\noindent

The shell script {\tt build\_os} created a
file system for Tunix using the Tunix utility {\tt mkfs}
which can be found in the subdirectory {\tt os/result}.
This file system is empty. The first thing you will
want to do is populate it with some directories and
files. This is simultaneously a first exercise in
using Tunix.


Move to the subdirectory {\tt os/result} and start
the Tunix system by executing the shell script
{\tt startup}. The driver process {\tt devtty}
for terminals must be started separately. This
is best done in a separate window (workstations)
or on a separate virtual terminal (Unix System V)
so as to keep the input and output of various
Tunix processes from getting mixed. Switch to
a different window and in the the subdirectory
{\tt os/result} enter the command
\begin{verbatim}
devtty -m2
\end{verbatim}


Now all parts of the Tunix OS are running. When
you switch back to the window in which the
kernel was started, you should see the first
few lines of the process table as reported
by the process manager PM in verbose mode.
You will see that only one process exists
with PID 5. This is essentially the {\tt init}
process but it can be used for doing
administrative work, as we will now see.


Switch to a new window or virtual terminal
and in the subdirectory {\tt os/result}
enter the command
\begin{verbatim}
user 5
\end{verbatim}
This starts a `user process' which regards
itself as having PID 5. It can now communicate
with the kernel. You can use these `user processes'
much as you would the shell, entering commands
like {\tt ls} and {\tt cd ...}. In addition
the user processes accept a number of commands
ordinary shells don't understand as we will see.
If you want to know which commands the user
process understands, enter the command {\tt help}
or else examine the code of {\tt user.c} in
the subdirectory {\tt os/kernel/source}.

When the prompt {\tt >} of the user process appears
you can enter the command {\tt ls}. The answer
will be `nothing', since your file system is
still empty. Now enter the following commands
to get things started.
\begin{verbatim}
mkdir bin
mkdir dev
mkdir etc
cd dev
mknod fs    060600 0
mknod fs1   060600 1
mknod rroot 020600 0
mknod tty   020622 512
\end{verbatim}
If you now enter the command {\tt ls} you should
get an answer that looks like this:
\begin{verbatim}
brw-------  1   1   1    0,  0 Mar 3 93 10:09  fs
brw-------  1   1   1    0,  1 Mar 3 93 10:10  fs1
crw-------  1   1   1    0,  0 Mar 3 93 10:10  rroot
crw--w--w-  1   1   1    2,  0 Mar 3 93 10:10  tty
\end{verbatim}
Now enter the command {\tt sync} to be sure your
changes are recorded permanently in the file system.


In principle you could fill your file system with
directories and files `by hand'. That is very
tedious. A quicker way is to use the Tunix
utility {\tt tree}. This program accepts a
Unix path as argument and makes an exact copy of
the Unix subdirectory specified by this path
in your Tunix root file system beneath the
subdirectory {\tt etc}. To do this the
Tunix kernel should be running of course.

\aaa
{\bf Example}\\
Suppose you have a Unix directory {\tt hooha}
which contains several subdirectories and each
of these contains several files. Start the
system with {\tt startup} and {\tt devtty -m2}.
Instead of starting the usual user process
with {\tt user} enter the command
\begin{verbatim}
tree hooha
\end{verbatim}
When {\tt tree} is finished, stop the kernel
again with {\tt shutdown}. Later with
your `user process' you can change
to the directory {\tt etc}. You will find
that it now has a subdirectory {\tt hooha}
and that {\tt hooha} has the same tree structure
as the original Unix directory. If some of
the files in this directory are ASCII files
you can list their contents with the command {\tt cat}.

\section{User processes}
\noindent

You can now test Tunix in `multiuser mode'. You start Tunix
in the manner described above. Execute the shell script
{\tt startup} in one window and enter the command
\begin{verbatim}
devtty -m2
\end{verbatim}
in a second window. In a third window enter the
command
\begin{verbatim}
user 5
\end{verbatim}
to start your {\tt init} process. This one should run
as long as the kernel is running. You may now generate
further processes by entering the command {\tt fork}
after the prompt in any `user process'. If you then
examine the process table printed by PM, you will
see that a new process has been created and you will
also see which PID it has been given. In a new window
(to keep output from being mixed) enter the
command
\begin{verbatim}
user n
\end{verbatim}
where {\tt n} is the PID of the newly created process.


You can tell which process is currently running --
i.e.\ `has the CPU' -- either by looking at the process
table or by seeing which user process has its prompt
visible. The running process has an {\tt O} in the
{\tt status} column of the process table. You should write commands
only at running processes. Writing commands at sleeping
or preempted processes has no effect.


\section{Debugging}
\noindent

The most difficult aspect of debugging has to do
with the communication protocol between the various
system and user processes: which process sent what
messages to which other process when? Here the
log files can be extremely helpful -- in fact,
indispensible. You can cause any of the system
processes to write a log file by calling
{\tt startup} with options. The options accepted
by {\tt startup} are
\begin{verbatim}
k h s f m i p
\end{verbatim}
For example the command
\begin{verbatim}
startup k f p
\end{verbatim}
will cause the kernel, FM and PM to write log files.
The option {\tt h} is for {\tt devhd}, the option
{\tt s} for {\tt devsw}, {\tt m} for MM and {\tt i}
for IPCM. The log file for the kernel is called
{\tt klog}, that of FM is called {\tt fmlog}, etc.


By starting a user process with a command like
\begin{verbatim}
user 5 -d
\end{verbatim}
you can cause it to write a log file as well. Its
log file will be called {\tt log5}.


These log files are not human readable. To look at them
we use the Tunix utility {\tt encore}. For example,
to examine the log produced by FM enter the command
\begin{verbatim}
encore fmlog
\end{verbatim}
The utility {\tt encore} writes one line for each message
recorded in the log giving the command, the recipient,
the sender and the first three arguments. All of this
is written to {\tt stdout}, so you are well advised
to pipe the output from {\tt encore} into {\tt more}
or else redirect it to a file, where you can inspect
it in a more leisurely fashion using an editor.


These log files tend to be very long, so you will need
a little practice in finding the place where something
went wrong. If you're lucky, it will be near the end
of the file.


\section{Closing remark}
\noindent

This is supposed to be software for people who want to
learn about operating systems by experimenting. So if
you run into any problems, prove that you're made of the
right stuff by fixing them yourself! Don't bombard me
with bug reports.


\end{document} 

